%& -shell-escape

\documentclass[]{beamer}

\mode<presentation>
{
%   \usetheme[blue,noshadow]{Trondheim}
%   \usetheme[blue,minimal]{Trondheim}
  \usetheme[blue,compress,numbers,nonav,noshadow]{Trondheim}
%   \usetheme[sand,compress,numbers,nonav,innovation]{Trondheim}
  % Some examples for the different options for beamerTrondheim

   %\usecolortheme{ntnuold}
  % try this if you encounter problems with the new ntnublue theme

  % \usefonttheme{professionalfonts}
  % Use only if using a font matching the conditions (see beamer docs)

  \usefonttheme[onlymath]{serif}
  % \useoutertheme{infolines}
  % or whatever

  \setbeamercovered{transparent}
  % or whatever (possibly just delete it)
}
\usepackage{ulem}
\usepackage{pdfsync}
\usepackage{color}
\usepackage{tikz}
\usepackage{etex}
\usetikzlibrary{shapes.arrows}
\usetikzlibrary{arrows,positioning,patterns,calc,shapes}
\usetikzlibrary{decorations.shapes,decorations.pathreplacing,decorations.markings}

\usepackage{bm}
% you only need this when using TikZ graphics


%\usepackage{multimedia}
\usepackage{movie15}
% you probably want to comment this out if not using multimedia elements
\usepackage{multirow}

%\usepackage[unicode]{hyperref}
%\usepackage[unicode]{hyperref}
\hypersetup{unicode}

\usepackage[english]{babel}
% or whatever

\usepackage[latin1]{inputenc}
% or whatever

\usepackage{mathptmx}
\usepackage{helvet}
\usepackage{courier}
\usepackage{amsmath}
 %\usepackage{arev}
% The non-standard packages arev and bera define fonts which look nicely for
% projection, you might want to try them instead of Times/Helvetica/Courier.
% Use the
\usepackage{pstcol}
\usepackage{url}
\usepackage[T1]{fontenc}
% Or whatever. Note that the encoding and the font should match. If T1
% does not look nice, try deleting the line with the fontenc.
\newgray{darkergray}{0.3}
\definecolor{Mygray}{gray}{0.75}


\tikzset{
    %Define standard arrow tip
    >=stealth',
    %Define style for boxes
    punkt/.style={
           rectangle,
           rounded corners,
           draw=black, very thick,
           text width=6.5em,
           minimum height=2em,
           text centered},
    % Define arrow style
    pil/.style={
           ->,
           thick,
           shorten <=2pt,
           shorten >=2pt,}
}



%New command shortcuts
\newcommand{\abs}[1]{\lvert#1\rvert}
\newcommand{\norm}[1]{\lVert#1\rVert}
\newcommand{\mx}[1]{\mathbf{\bm{#1}}} % Matrix command
\newcommand{\vc}[1]{\mathbf{\bm{#1}}}
\newcommand{\Rn}{\mathbb{R}^n}
\newcommand{\Rmn}{\mathbb{R}^{m \times n}}
\newcommand{\Lp}{\mathcal{L}_p}
\newcommand{\Li}[1]{\mathcal{L}_{#1}}
\newcommand{\xd}{\dot{x}}
\newcommand{\Dset}{\mathbb{D}}
\newcommand{\Ri}[1]{\mathbb{R}^{#1}}

\title[Lecture 1: Introduction] %(optional, use only with long paper titles)
{TK8103 Advanced Nonlinear Control}

%\author{Lecture 1}
\institute{}

 \subtitle
 {Lecture 1: Introduction}

%\author{J\"{o}rg Cassens}

% Till Tantau\author{{1} \and
% J\"{o}rg Cassens\inst{2}

%\institute[NTNU] % (optional, but mostly needed)
%{Norwegian University of Science and Technology}
% - Use the \inst command only if there are several affiliations.
% - Keep it simple, no one is interested in your street address.


\date[] % (optional, should be abbreviation of conference name)
{}
% - Either use conference name or its abbreviation.
% - Not really informative to the audience, more for people (including
%   yourself) who are reading the slides online

\subject{Beamer}
% This is only inserted into the PDF information catalog. Can be left
% out.

% \AtBeginSubsection[]
% {
%   \begin{frame}<beamer>
%     \frametitle{Outline}
%     \tableofcontents[currentsection,currentsubsection]
%   \end{frame}
% }
% Use this if you do want the table of contents to pop up at
% the beginning of each subsection.

% \pgfdeclareimage[height=2em,interpolate=true]{ntnulogotext}{foo}
% If you want to include a different logo on the title page
% only (e.g. a combined logo of different institutions), you
% can use this command.

\begin{document}

\pgfdeclareimage[width=.6\textwidth]{ekstern}{./figures/nonregistered.jpg}
\pgfdeclareimage[width=.25\textwidth]{smiley}{./figures/Smiley.jpg}
\pgfdeclareimage[width=.25\textwidth]{Arnfinn}{./figures/Arnfinn.jpg}

%\maketitle
% \compressedtitle
% You can use \maketitle to create the titlepage,
% or \compressedtitle to create a more compact titlepage
% with the look of the other pages in compress style

 \begin{frame}
  %\begin{centering}
  %{\Big TTK4105 Nonlinear Control Systems}\\
  %Lecture 1
  %\end{centering}
  \titlepage
 \end{frame}
% Or you can call the titlepage command in a frame environment

\begin{frame}<beamer>[allowframebreaks]
  \frametitle{Outline}
  \tableofcontents
\end{frame}

%%%%%%%%%%%%%%%%%%%%%%%%%%%%%%%%%%%%%%%%%%%%%%%%%%%%%%%%%%%%%%%%%%
\section{Today's goals}
%%%%%%%%%%%%%%%%%%%%%%%%%%%%%%%%%%%%%%%%%%%%%%%%%%%%%%%%%%%%%%%%%%%%

\subsection{Today's goals}
\begin{frame}[t]
	\frametitle{\color{white} Today's goals}
	\begin{block}{After this course you should...}
		\begin{itemize}
		\item Know more about the backstepping technique and how it can be applied to nonlinear systems
			\begin{itemize}
			\item Or maybe you will just know how to make slides 
			\item Or maybe \alert{not}?
			\end{itemize}
			
		\item Goal 1: item 
			\begin{itemize}
			\item Goal 1.1: subitem
			\end{itemize}
		\end{itemize}
	\end{block}
\end{frame}


%%%%%%%%%%%%%%%%%%%%%%%%%%%%%%%%%%%%%%%%%%%%%%%%%%%%%%%%%%%%%%%%%%%%%

\begin{frame}[t]
\frametitle{Literature}
    \begin{block}{Today's lecture is based on}
        \begin{tabular}{ll}
        Khalil & \!\!{\bf Section 13.4 Backstepping} \\
        & - Theory \\
        & - Examples \\
        \end{tabular}
    \end{block}
\end{frame}


%%%%%%%%%%%%%%%%%%%%%%%%%%%%%%%%%%%
\section{Theory}
%%%%%%%%%%%%%%%%%%%%%%%%%%%%%%%%%%%
\begin{frame}[t]
	\frametitle{Theory}
	FIGURE!
	\begin{tikzpicture}[scale=1.5]
	\coordinate (o) at (0,0);
	\fill (0,0) circle (2pt);
	%	\draw[rotate=-45, red] (0,0) circle (0.3 and 0.5);
	%	\draw[rotate=0] (-45:0.3) -- (2,-2) node[right] {$\Omega_2=\{x \in B_r: W_2(x) \leq c\}$};
	%	\draw[rotate=0, green] (0,0) circle (0.5 and 1);
	%	\draw[rotate=0] (0.5,0) -- (2,-1) node[right] {$\Omega_{t,c} = \{x \in B_r: V(t,x) \leq c\}$};
		\draw[rotate=-45, red] (0,0) circle (0.9 and 1.4   );
		\draw (0:1.0) -- (2.5,0) node[right] {$\Omega_1=\{x \in B_r: W_1(x) \leq c\}$};
	\draw[rotate=-45, blue] (0,0) circle (1.5 and 1.5);
	\draw (25:1.5) -- (2,1) node[right] {$B_r$};
	\draw[rounded corners=1cm,rotate=-45,scale=0.8] (-2,-2) rectangle (2,2);
	\node at (1.2,1.6) {$\Dset$};
	\end{tikzpicture}
\end{frame}

%%%%%%%%%%%%%%%%%%%%%%%%%%%%%%%%%%%
\section{Questions}
%%%%%%%%%%%%%%%%%%%%%%%%%%%%%%%%%%%
\subsection{Question 1}
\begin{frame}[t]
	\frametitle{Question 1}
	
	\begin{alertblock}{Question 1}
		Questions?
		\begin{equation*}
			No?\ Great!
		\end{equation*}
	\end{alertblock}
\end{frame}	


%%%%%%%%%%%%%%%%%%%%%%%%%%%%%%%%%%%
\section{Examples}
%%%%%%%%%%%%%%%%%%%%%%%%%%%%%%%%%%%
\begin{frame}[t]
	\frametitle{Example 1: Integrator backstepping}
	
	\begin{exampleblock}{Example}
		$\displaystyle V(x) =\frac{x_1^2}{(1 + x_1^2)} + x_2^2$
		
		
	\end{exampleblock}
\end{frame}



	
	
\end{document} 